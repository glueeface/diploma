\sectionCenteredToc{ВВЕДЕНИЕ}
\label{sec:intro}

%Во введении кратко указывается, чему посвящен дипломный проект, делается обзор состояния предметной области,
%формулируется общая цель разра-ботки и ее место среди известных отечественных и зарубежных аналогов.

С развитием информационных технологий и финансовых рынков трейдинг стал более доступным и привлекательным для широкого круга инвесторов.
В связи с этим возрос спрос на эффективные инструменты и платформы, которые обеспечивают быстрый и удобный доступ к торговым операциям на различных рынках.
В этом контексте создание и развитие трейдинговых платформ становится ключевым направлением в индустрии финансовых технологий.

Данная работа посвящена разработке и реализации трейдинговой платформы, предназначенной для облегчения процесса торговли акциями, валютой, криптовалютой и другими финансовыми инструментами.
Основной функционал будет направлен на автоматизацию процессов торговли.
Основной задачей будет являться создание современного и интуитивно понятного функционала, способного удовлетворить потребности как опытных, так и начинающих трейдеров.
%В рамках данной работы будут рассмотрены основные аспекты разработки трейдинговой платформы, включая выбор архитектурных принципов, функциональных возможностей и безопасности.

Проект будет реализовывать систему, которая позволяет пользователю торговать на бирже автоматически.
Пользователь указывает определенную позицию, цену за которую он готов её купить, а также количество для покупки.
Также работает и автоматическая продажа: пользователь указывает позицию, цену по которой он готов её продать и количество для продажи.

Будет реализовано два сервиса, написанных с использованием фреймворков языка программирования Python.
Основной микросервис будет основан на Django REST Framework и будеть отвечать за обработку запросов пользователей и выполнение соответствующих действий на бирже.
К таким действиям будут относиться:
\begin{enumerate_num}
    \item регистрация пользователей;
    \item пополнение счетов пользователей;
    \item покупка и продажа валюты;
    \item уведомление пользователей о событиях;
\end{enumerate_num}

Второй микросервис будет основан на FastAPI и будет использоваться для получения данных о курсе акций и криптовалют в реальном времени, а также передачи этих данных на основной микросервис.

В соответствии с поставленной целью были определены следующие задачи:
\begin{enumerate_num}
    \item Исследование и анализ рынка.
    \item Проектирование модулей трейдинговой платформы.
    \item Разработка модулей трейдинговой платформы.
    \item Тестирование работоспособности трейдинговой платформы.
    \item Расчет экономических показателей дипломного проекта.
    \item Написание руководства пользователя.
\end{enumerate_num}

%Краткая характеристика проекта:
%Проект реализует систему, которая позволяет пользователю торговать на бирже автоматически.
%Пользователь указывает определенную позицию, цену за которую он готов её купить, а также количество для покупки.
%Также работает и автоматическая продажа: пользователь указывает позицию, цену по которой он готов её продать и количество для продажи.
%Проект состоит из двух микросервисов, написанных с использованием фреймворков языка программирования Python.
%Основной микросервис написан на Django REST Framework и отвечает за обработку запросов от пользователя и выполнение соответствующих действий на бирже.
%Второй микросервис написан на FastAPI и используется для получения данных о курсе акций и криптовалют в реальном времени и передачи этих данных на основной микросервис.

