\section{\texorpdfstring{\MakeUppercase \economicalPartName}{\economicalPartName}}

% Begin Calculations

\subsection{Краткая характеристика программного средстваx}

%Необходимо сформировать представление о программном средстве как о товаре, давего краткую характеристику:
%–указать цель разработки, область применения, какие задачи решает, ка-кие функции выполняет;
%–указать организацию-разработчика и потенциальных покупателей программного средства(профиль целевой аудитории);
%–подтвердить наличие актуальной потребности в разрабатываемом программном средстве, в том числе перечислить существующие на рынке
%конкурентные аналоги.
%На основе маркетинговых исследований потребительского спроса необходимо обосновать прогнозируемый годовой объем продаж в течение
%расчетного периода и прогнозируемую цену копии (лицензии).
%Также следует указать (с кратким  обоснованием)  предполагаемую  модель  монетизации  программного продукта:
%платное  приложение  (Paid  App),
%подписка  (Subscription),
%реклама внутри приложения (In-app Advertising),
%бесплатная базовая и платная расширен-ная версии (Freemium),
%транзакционная модель, покупки внутри приложения (In-app Purchases) и т.п.
%Стоит определить предполагаемые каналы продаж программного продукта (например,через  платформу  Google  Play или  AppStore).
%Крайне  желательно кратко сформулировать стратегию продвижения программного продукта,
%а также указать, что в результате организация-разработчик получит экономический эффект в виде прироста чистой прибыли,
%полученной от их реализации.

Разработанный дипломный проект является трейдинговой платформой и решает задачу своевременного проведения транзакций трейдерами.
Основная область применения продукта -- профессиональными трейдерами, инвесторами и розничными клиентами для осуществления торговли на финансовых рынках.

Продукт имеет следующие функции:
\begin{itemize}
  \item пополнение и вывод счёта пользователя;
  \item торговля валютами, криптовалютами и акциями;
  \item создание автоматических офферов на покупку/продажу валют, криптовалют и акций;
  \item просмотр и анализ рынка валют;
  \item просмотр и анализ пользовательских транзакций.
\end{itemize}

Целевой аудиторией будут выступать пользователи, которые заинтересованы в трейдинге и хотят начать свой путь в трейдинге.
Помимо начинающих трейдеров, часть аудитории будет состоять из профессиональных трейдеров или инвесторов,
которые хотят упростить процесс торговли.

Важность этого программного средства обусловлена растущим спросом на автоматизацию процессов торговли на финансовой бирже и необходимостью
инструментов, имеющих достаточный функционал для полноценной торговли на бирже.

%Потенциальные экономические выгоды от использования данного программного средства включают повышение количества транзакций путем их автоматизации и
%и пониженная стоимость программного обеспечения на ранних этапах использования, что может привести к увеличению опыта и прибыли для начинающего трейдера.

Этот продукт разрабатывается для массового рынка, что предполагает его широкое использование и достаточный спрос для успешной реализации
на рынке информационных технологий.
Помимо этого количество пользователей будет расти не только по мере роста популярности разрабатываемого продукта, но и по мере роста самой сферы трейдинга.

\subsection{Расчет инвестиций в разработку программного средства}

\subsubsection{Расчет затрат на основную заработную плату разработчиков}

%Затраты на разработку высчитываются исходя из:
%\begin{itemize}
%    \item количества исполнителей;
%    \item объема работ исполнителей;
%    \item размера премии исполнителей;
%\end{itemize}
Затраты на разработку высчитываются исходя из количества исполнителей, объема работ исполнителей, размера премии исполнителей.

Расчет затрат на основную заработную плату можно произвести по формуле (4.1).

\begin{equation}
  \label{eq:econ:Zo}
  \text{З}_\text{о} = \text{К}_\text{пр} \cdot
    \sum_{i = 1}^{n} \text{З}_{\text{ч}i} \cdot t_i,
\end{equation}
\begin{explanationx}
  \item[где] $ \text{К}_\text{пр} $ -- коэффициент премий;
  \item $ n $ -- количество категорий исполнителей, занятых разработкой
  программного средства;
  \item $ \text{З}_{\text{ч}i} $ -- часовая заработная плата исполнителя $ i $-й категории, \rub;
  \item $ t_i $ -- трудоемкость работ исполнителя $ i $-й категории, ч.
\end{explanationx}

Размер месячной заработной платы исполнителя каждой категории соответствует сложившемуся на рынке труда размеру заработной платы для данных категорий.
В качестве официального информационного источника по заработной плате был использован отечественный портал.

Расчет затрат на основную заработную плату команды разработчиков представлен в таблице 4.1.
%~\ref{table:econ:calc_zar_plata}.
В качестве опорной выступает медианная заработная плата.
Часовая заработная плата каждого исполнителя определяется путем деления его месячной заработной платы на количество рабочих часов в месяце – 168 часов.
Размер премии составляет 40\% от размера основной заработной платы.

\FPeval{\valPremiaPercent}{40}
\FPeval{\valKpr}{round(1 + \valPremiaPercent / 100, \configRoundSigns)}
\FPeval{\valtMonth}{168}

\FPeval{\valZchVedProger}{round(4500, \configRoundSigns)}
\FPeval{\valHourVedProger}{clip(\valZchVedProger / \valtMonth)}
\FPeval{\valHourVedProgerPrint}{round(\valHourVedProger, \configRoundSigns)}
\FPeval{\valtVedProger}{round(\valtMonth * 2, 0)}
\FPeval{\valTotalVedProger}{round(\valHourVedProger * \valtVedProger, \configRoundSigns)}

\FPeval{\valZchProger}{round(1955, \configRoundSigns)}
\FPeval{\valHourProger}{clip(\valZchProger / \valtMonth)}
\FPeval{\valHourProgerPrint}{round(\valHourProger, \configRoundSigns)}
\FPeval{\valtProger}{round(\valtMonth * 2, 0)}
\FPeval{\valTotalProger}{round(\valHourProger * \valtProger, \configRoundSigns)}

\FPeval{\valZchTester}{round(2600, \configRoundSigns)}
\FPeval{\valHourTester}{clip(\valZchTester / \valtMonth)}
\FPeval{\valHourTesterPrint}{round(\valHourTester, \configRoundSigns)}
\FPeval{\valtTester}{round(\valtMonth / 2, 0)}
\FPeval{\valTotalTester}{round(\valHourTester * \valtTester, \configRoundSigns)}

\FPeval{\valZchBA}{round(4335, \configRoundSigns)}
\FPeval{\valHourBA}{clip(\valZchBA / \valtMonth)}
\FPeval{\valHourBAPrint}{round(\valHourBA, \configRoundSigns)}
\FPeval{\valtBA}{\valtMonth}
\FPeval{\valTotalBA}{round(\valHourBA * \valtBA, \configRoundSigns)}

\FPeval{\valTotal}{round(\valTotalProger + \valTotalTester + \valTotalVedProger + \valTotalBA, \configRoundSigns)}
\FPeval{\valPremiaSum}{round(\valTotal * \valPremiaPercent / 100, \configRoundSigns)}
\FPeval{\valZo}{round(\valTotal + \valPremiaSum, \configRoundSigns)}

\begin{table}[ht]
  \caption{Расчет затрат на основную заработную плату разработчиков}
  \label{table:econ:calc_zar_plata}
  \begin{tabular}{| >{\raggedright}m{0.20\textwidth}
                  | >{\centering}m{0.18\textwidth}
                  | >{\centering}m{0.18\textwidth}
                  | >{\centering}m{0.18\textwidth}
                  | >{\centering\arraybackslash}m{0.127\textwidth}|}
      \hline
      \centering Категория исполнителя
      & Месячная заработная плата, \rub
      & Часовая заработная плата, \rub
      & Трудоемкость работ, ч
      & Итого, \rub \\

      \hline
      Ведущий-программист
      & \num\valZchVedProger
      & \num\valHourVedProgerPrint
      & \num\valtVedProger
      & \num\valTotalVedProger
      \\

      \hline
      Инженер-программист
      & \num\valZchProger
      & \num\valHourProgerPrint
      & \num\valtProger
      & \num\valTotalProger
      \\

      \hline
      Инженер-тестировщик
      & \num\valZchTester
      & \num\valHourTesterPrint
      & \num\valtTester
      & \num\valTotalTester
      \\

      \hline
      Бизнес-аналитик
      & \num\valZchBA
      & \num\valHourBAPrint
      & \num\valtBA
      & \num\valTotalBA
      \\

      \hline
      \multicolumn{4}{|l|}{Итого}
      & \num\valTotal
      \\

      \hline
      \multicolumn{4}{|l|}{Премия ($ \num\valPremiaPercent \ \% $)}
      & \num\valPremiaSum
      \\

      \hline
      \multicolumn{4}{|l|}{Всего затраты на основную заработную плату разработчиков}
      & \num\valZo
      \\

      \hline
  \end{tabular}
\end{table}

\fixTableSectionSpace

\subsubsection{Расчет затрат на дополнительную заработную плату разработчиков}

\FPeval{\valNdPercent}{15}

Для расчета затрат на дополнительную заработную плату разработчиков воспользуемся
следующей формулой:
\begin{equation}
  \label{eq:econ:Zd}
  \text{З}_\text{д} = \frac{\text{З}_\text{о} \cdot \text{Н}_\text{д}}
    {100},
\end{equation}
\begin{explanationx}
  \item[где] $ \text{Н}_\text{д} $ -- норматив дополнительной заработной платы.
\end{explanationx}

Будем считать, что значение норматива дополнительной заработной платы составляет $ \num\valNdPercent \ \% $.

\FPeval{\valZd}{round(\valZo * \valNdPercent / 100, \configRoundSigns)}

\subsubsection{Расчет отчислений на социальные нужды}

\FPeval{\valNSotsPercent}{35}

Размер отчислений на социальные нужды определяется ставкой отчислений, которая
в соответствии с действующим законодательством по состоянию на \econCalcDate\
составляет $ \num\valNSotsPercent \ \% $. Найдем размер отчислений по формуле:

\begin{equation}
  \label{eq:econ:RSots}
  \text{Р}_\text{соц} = \frac{(\text{З}_\text{о} + \text{З}_\text{д}) \cdot \text{Н}_\text{соц}}
    {100},
\end{equation}
\begin{explanationx}
  \item[где] $ \text{Н}_\text{соц} $ -- ставка отчислений на социальные нужды.
\end{explanationx}

\FPeval{\valRSots}{round((\valZo + \valZd) * \valNSotsPercent / 100, \configRoundSigns)}

\subsubsection{Расчет прочих расходов}

\FPeval{\valNPrPercent}{30}

Прочие расходы рассчитываются с учетом норматива прочих расходов.
Примем значение норматива равным $ \num\valNPrPercent \ \% $.
Используя это значение рассчитаем прочие расходы по формуле:

\begin{equation}
  \label{eq:econ:RPr}
  \text{Р}_\text{пр} = \frac{\text{З}_\text{о} \cdot \text{Н}_\text{пр}}
    {100},
\end{equation}
\begin{explanationx}
  \item[где] $ \text{Н}_\text{пр} $ -- норматив прочих расходов.
\end{explanationx}

\FPeval{\valRPr}{round(\valZo * \valNPrPercent / 100, \configRoundSigns)}

\subsubsection{Расчет расходов на реализацию}

\FPeval{\valRPercent}{5}

Для разрабатываемого приложения норматив расходов на реализацию составляет $ \num\valRPercent \ \% $.
Под расходами на реализацию понимают выраженные в денежной форме затраты материальных, трудовых и других видов ресурсов торговых организаций по доведению продукта до конечного потребителя, которые рассчитываются по формуле:

\begin{equation}
  \label{eq:econ:R}
  \text{Р}_\text{р} = \frac{\text{З}_\text{о} \cdot \text{Н}_\text{р}}
    {100},
\end{equation}
\begin{explanationx}
  \item[где] $ \text{Н}_\text{р} $ -- норматив расходов на реализацию.
\end{explanationx}

\FPeval{\valR}{round(\valZo * \valRPercent / 100, \configRoundSigns)}

\subsubsection{Расчет общей суммы затрат}

Определим общую сумму затрат как сумму ранее вычисленных расходов: на основную
заработную плату, дополнительную заработную плату, отчислений на социальные нужды и
прочие расходы.
Для определения этого показателя используется следующая формула:
\fixTableSectionSpace
\begin{equation}
  \label{eq:econ:Zr}
  \text{З}_\text{р} = \text{З}_\text{о} + \text{З}_\text{д}
    + \text{Р}_\text{соц} + \text{Р}_\text{пр} + \text{Р}_\text{р}.
\end{equation}

\FPeval{\valZr}{round(\valZo + \valZd + \valRSots + \valRPr , \configRoundSigns)}

% \subsubsection{Расчет плановой прибыли}

% \FPeval{\valRPsPercent}{37}

% Плановая прибыль рассчитывается как процент от общей суммы затрат, называемый
% уровнем рентабельности. Определим уровень рентабельности
% в $ \num\valRPsPercent \ \% $. Зная это значение, можно произвести расчет плановой
% прибыли по формуле:

% \begin{equation}
%   \label{eq:econ:PPs}
%   \text{П}_\text{пс} = \frac{\text{З}_\text{р} \cdot \text{Р}_\text{пс}}
%     {100},
% \end{equation}
% \begin{explanationx}
%   \item[где] $ \text{Р}_\text{пс} $ -- рентабельность затрат на разработку программного средства.
% \end{explanationx}

% \FPeval{\valPPs}{round(\valZr * \valRPsPercent / 100, \configRoundSigns)}

% \subsubsection{Расчет отпускной цены программного средства}

% Отпускная цена с учетом налога на добавочную стоимость соответствует размеру
% инвестиций, вкладываемых заказчиком для разработки программного средства.

% Определяется отпускная цена как сумма расходов и плановой прибыли.
% Найти значение можно по формуле:

% \begin{equation}
%   \label{eq:econ:TsPs}
%   \text{Ц}_\text{пс} = \text{З}_\text{р} + \text{П}_\text{пс}.
% \end{equation}

% \FPeval{\valTsPs}{round(\valZr + \valPPs, \configRoundSigns)}

С использованием ранее приведенных формул найдем значение затрат, определим общую сумму затрат на разработку в таблице 4.2.
%~\ref{table:econ:calc_invest_development}.

%
%  \begin{tabular}{| >{\raggedright}m{0.35\textwidth}
%                  | >{\centering}m{0.41\textwidth}
%                  | >{\centering\arraybackslash}m{0.16\textwidth}|}

\begin{table}[ht]
  \caption{Расчет инвестиций в разработку программного средства}
  \label{table:econ:calc_invest_development}
  \begin{tabular}{| >{\raggedright}m{0.325\textwidth}
                  | >{\centering}m{0.425\textwidth}
                  | >{\centering\arraybackslash}m{0.16\textwidth}|}
      \hline
      \centering Наименование статьи затрат
      & Расчет по формуле
      & Значение, \rub
      \\

      \hline
      Основная заработная плата разработчиков
      & см. таблицу 4.1 %~\ref{table:econ:calc_zar_plata}
      & \num\valZo
      \\

      \hline
      Дополнительная заработная плата разработчиков
%      & $ \text{З}_\text{д} = \frac{\num\valZo \cdot \num\valNdPercent}{100} $
%      & \num\valZd
      & \vspace{0.5em} $ \scalemath{1}{\text{З}_\text{д} = \frac{\num\valZo \cdot \num\valNdPercent}{100}} $ \vspace{0.5em}
      & \num\valZd
      \\

      \hline
      Отчисления на социальные нужды
%      & $ \text{Р}_\text{соц} = \frac{(\num\valZo + \num\valZd) \cdot \num\valNSotsPercent}{100} $
%      & \num\valRSots
      & \vspace{0.5em} $ \scalemath{1}{\text{Р}_\text{соц} = \frac{(\num\valZo + \num\valZd) \cdot \num\valNSotsPercent}{100}} $ \vspace{0.2em}
      & \num\valRSots
      \\

      \hline
      Прочие расходы
%      & \vspace{0.5em} $ \text{Р}_\text{пр} = \frac{\num\valZo \cdot \num\valNPrPercent}{100} $ \vspace{0.5em}
%      & \num\valRPr
      &  \vspace{0.5em} $ \scalemath{1}{\text{Р}_\text{пр} = \frac{\num\valZo \cdot \num\valNPrPercent}{100}} $ \vspace{0.5em}
      & \num\valRPr
      \\

      \hline
      Расходы на реализацию
      & \vspace{0.5em} $ \scalemath{1}{\text{Р}_\text{р} = \frac{\num\valZo \cdot \num\valRPercent}{100}} $ \vspace{0.5em}
      & \num\valR
      \\

      \hline
      Общая сумма затрат на разработку
      & $ \text{З}_\text{р} = \num\valZo + \num\valZd + \num\valRSots + \num\valRPr $
      & \num\valZr
      \\

      % \hline
      % 6. Плановая прибыль, включаемая в цену программного средства
      % & $ \text{П}_\text{пс} = \frac{\num\valZr \cdot \num\valRPsPercent}{100} $
      % & \num\valPPs
      % \\

      % \hline
      % 7. Отпускная цена программного средства
      % & $ \text{Ц}_\text{пс} = \num\valZr + \num\valPPs $
      % & \num\valTsPs
      % \\

      \hline
  \end{tabular}
\end{table}

\fixTableSectionSpace

\subsection{Расчет экономического эффекта от реализации программного средства на рынке}

\subsubsection{Расчет результата для организации-разработчика}

Экономический эффект организации-разработчика программного средства представляет собой прирост чистой прибыли от его продажи на рынке
потребителям, величина которого зависит от объема продаж, цены реализации и затрат на разработку программного средства.

\begin{equation}
  \label{eq:econ:deltaPCh}
  \Delta \text{П}_\text{ч} = \bigl(\text{Ц}_\text{отп} \cdot \text{N} - \text{НДС} \bigr) \cdot \text{Р}_\text{пр} \cdot \biggl( 1 - \frac{\text{Н}_\text{п}}{100} \biggr),
\end{equation}
\begin{explanationx}
  \item[где] $ \text{Н}_\text{п} $ -- ставка налога на прибыль согласно действующему законодательству, \%;
  \item $ \text{Ц}_\text{отп} $ -- отпускная цена копии (лицензии) программного средства, р.;
  \item $ \text{N} $ -- количество копий (лицензий) программного средства, реализуемое за год, шт.;
  \item $ \text{Р}_\text{пр} $ -- рентабельность продаж копий (лицензий), \%.
\end{explanationx}

\FPeval{\valCopyPrice}{70}
\FPeval{\valCopyCount}{4000}
\FPeval{\valCopyRent}{30}

Изучив рынок трейдеров и инвесторов, можно сделать вывод, что в СНГ по данным litefinance насчитывается около 1 миллиона трейдеров.
Количество профессиональных трейдеров, активно ведущих работу на бирже составляет лишь около 100 тыс.
Предположим, что 5\% из тех, что не ведут активную работу на бирже, могли бы быть заинтересованы в предлагаемом продукте, что составляет около 45000 пользователей.
Также предположим 8--10\% из них будут готовы приобрести продукцию для использования в первый год продаж.

На основании данных предположений, прогнозируемый годовой объем продаж программного модуля колеблется между 3 и 5 тысячами единиц.
Если выбрать среднее значение в этом диапазоне, ожидаемый объем продаж составит около $ \num\valCopyCount $ копий за год.

Анализ рыночных цен на аналогичные программные продукты в области трейдинга показывает, что стоимость продвинутой годовой подписки на трейдинговых платформах варьируется от 100 до 400 рублей.
В то же время, продукт выделяется на фоне конкурентов за счет наличия обширного функционала,
предлагая пользователям передовые возможности для эффективного и удобного трейдинга.

С учетом существующего ценового диапазона для подобных программных продуктов,
а также принимая во внимание уникальные преимущества нашего решения, предлагается установить цену продукта немного ниже среднего рыночного уровня.
Рассматривая ценовой диапазон от 45 до 85 рублей, продукт может быть оценен в $ \num\valCopyPrice $ рублей за копию,
что делает его более доступным среди конкурентов, но одновременно отражает его ценность на рынке.

В итоге, определяя цену в $ \num\valCopyPrice $ рублей за копию, мы подчеркиваем высокое качество и эксклюзивность нашего программного
модуля, делая его привлекательным выбором для целевой аудитории в сфере трейдинга.
%Эта цена соответствует как потребностям рынка, так и экономическим целям нашего продукта, обеспечивая его конкурентоспособность и востребованность.

Исходя из полученных данных можно произвести оставшиеся вычисления.

Налог на добавленную стоимость определяется по формуле:

\begin{equation}
  \label{eq:econ:nds}
    \text{НДС} = \frac{\text{Ц}_\text{отп} \cdot \text{N} \cdot \text{Н}_\text{дс}}
    {100 \% + \text{Н}_\text{дс}},
\end{equation}
\begin{explanationx}
  \item[где] $ \text{Н}_\text{дс} $ -- ставка налога на добавленную стоимость в соответствии с
действующим законодательством, \%;
\end{explanationx}

\FPeval{\valNdsPercent}{20}
\FPeval{\valNds}{round((\valCopyPrice * \valCopyCount * \valNdsPercent)/(100 + \valNdsPercent), \configRoundSigns)}


По состоянию на \econCalcDate, ставка налога на добавленную стоимость составляет $ \num\valNdsPercent \ \% $.

Вычисление налога на добавленную стоимость:

\begin{equation}
  \label{eq:econ:ndsCalc}
    \text{НДС} = \frac{\num\valCopyPrice \cdot \num\valCopyCount \cdot \num\valNdsPercent}
    {100 + \num\valNdsPercent} = \num\valNds \rubFormula
\end{equation}

\FPeval{\valNPPercent}{20}
\FPeval{\valDeltaPCh}{round((\valCopyPrice * \valCopyCount - \valNds) * (\valCopyRent / 100) * (1 - \valNPPercent / 100), \configRoundSigns)}

По состоянию на \econCalcDate, ставка налога на прибыль составляет $ \num\valNPPercent \ \% $, а рентабельность $ \num\valCopyRent \ \% $ .
Используя данные значения, найдем прирост чистой прибыли по формуле~(4.7):
%\ref{eq:econ:deltaPCh}
%\vspace{-1em}
\begin{equation}
  \label{eq:econ:deltaPChCalc}
  \Delta \text{П}_\text{ч} =  \bigl(\num\valCopyPrice \cdot \num\valCopyCount - \num\valNds \bigr) \cdot \frac{\num\valCopyRent}{100} \cdot \biggl( 1 - \frac{\num\valNPPercent}{100} \biggr) = \num\valDeltaPCh \rubFormula
\end{equation}

\fixTableSectionSpace

\subsection{Расчет показателей экономической эффективности разработки и реализации программного средства на рынке}

Для оценки экономической эффективности разработки и реализации программного средства на рынке необходимо сравнить сумму инвестиций в его разработку и полученный годовой прирост чистой прибыли.
Из вышеприведенных расчетов видно, что сумма инвестиций в разработку, равная $ \num\valZr \rubFormula$  меньше, чем годовой прирост чистой прибыли, равный $ \num\valDeltaPCh \rubFormula$ Из этого можно сделать вывод, что инвестиции окупятся меньше, чем через год.

Исходя из этого оценка экономической эффективности инвестиций в разработку
программного средства осуществляется с помощью расчета рентабельности инвестиций (Return on Investment, ROI),
которая рассчитывается по формуле:

\begin{equation}
  \label{eq:econ:Ri}
  \text{ROI} = \frac{\Delta \text{П}_\text{ч} - \text{З}_\text{р}}{\text{З}_\text{р}}
    \cdot 100 \ \%.
\end{equation}
\begin{explanationx}
  \item[где] $ \Delta \text{П}_\text{ч} $ -- прироста чистой прибыли, р.;
  \item[где] $ \text{З}_\text{р} $ -- полная сумма затрат на разработку программного обеспечения, р.
\end{explanationx}

\FPeval{\valRi}{round((\valDeltaPCh - \valZr) / \valZr * 100, \configPercentRoundSigns)}

Используя ранее определенные значения, найдем значение рентабельности инвестиций по формуле~(4.11):
%\ref{eq:econ:Ri}

\begin{equation}
  \label{eq:econ:RiCalc}
  \text{ROI} = \frac{\num\valDeltaPCh - \num\valZr}{\num\valZr}
    \cdot 100 = \num\valRi \ \%.
\end{equation}

\fixTableSectionSpace

\subsection{Вывод об экономической эффективности}

В результате технико-экономического обоснования разработки и
реализации трейдинговой платформы были получены следующие результаты:
общие затраты на разработку программного продукта составили $ \num\valZr $ рублей,
годовой прирост прибыли равен $ \num\valDeltaPCh $ рублей,
рентабельность инвестиций $ \num\valRi $ \%.

На основании расчетов можно сделать вывод, что разработка и реализация программного продукта является
эффективным вложением инвестиций, так как проект окупится менее, чем за год и рентабельность выше ставки Национального Банка по депозитам, составляющую 10,29 \% на \econCalcDate

% \subsubsection{Расчет результата для организации-заказчика}

% Разрабатываемое программное средство позволяет сэкономить на заработной плате
% и начислениях на заработную плату сотрудников за счет снижения трудоемкости работ.

% Для расчета экономии на заработной плате воспользуемся формулой:

% \begin{equation}
%   \label{eq:econ:eZP}
%   \text{Э}_\text{з.п} = \text{К}_\text{пр} \cdot
%     \bigl(t_\text{р}^\text{без п.с} - t_\text{р}^\text{с п.с} \bigr) \cdot
%     \text{Т}_\text{ч} \cdot N_\text{п} \cdot
%     \biggl( 1 + \frac{\text{Н}_\text{д}}{100} \biggr) \cdot
%     \biggl( 1 + \frac{\text{Н}_\text{соц}}{100} \biggr),
% \end{equation}
% \begin{explanationx}
%   \item[где] $ \text{К}_\text{пр} $ -- коэффициент премий;
%   \item $ t_\text{р}^\text{без п.с} $ -- трудоемкость выполнения работ сотрудниками до внедрения программного средства, ч;
%   \item $ t_\text{р}^\text{с п.с} $ -- трудоемкость выполнения работ сотрудниками после внедрения программного средства, ч;
%   \item $ \text{T}_\text{ч} $ -- часовой оклад (часовая тарифная ставка) сотрудника,
%   использующего программное средство, \rub;
%   \item $ N_\text{п} $ -- плановый объем работ, выполняемых сотрудником.
% \end{explanationx}

% \FPeval{\valTargetKPr}{round(1.75, \configRoundSigns)}
% \FPeval{\valTargetTCh}{round(5.36, \configRoundSigns)}
% \FPeval{\valTargetNP}{1}
% \FPeval{\valTargetTBefore}{3000}
% \FPeval{\valTargetTAfter}{2000}
% \FPeval{\valEZP}{round(\valTargetKPr * (\valTargetTBefore - \valTargetTAfter) *
%   \valTargetTCh * \valTargetNP * (1 + \valNdPercent / 100) *
%   (1 + \valNSotsPercent / 100), \configRoundSigns)}

% Определим коэффициент премий у организации-заказчика равным $ \text{К}_\text{пр} =
% \num\valTargetKPr $.

% Часовой оклад будем считать равным $ \text{T}_\text{ч} =
% \num\valTargetTCh \rubFormula $

% Плановый объем работ, выполняемых сотрудником, примем равным
% $ N_\text{п} = \num\valTargetNP $.

% Трудоемкость выполнения работ сотрудниками до внедрения примем равным
% $ t_\text{р}^\text{без п.с} = \num\valTargetTBefore \text{ ч} $, а после --
% $ t_\text{р}^\text{с п.с} = \num\valTargetTAfter \text{ ч} $.

% Используя найденные значения, определим экономию для организации-заказчика по
% формуле~(\ref{eq:econ:eZP}):

% \begin{equation}
%   \label{eq:econ:eZPCalc}
%   \begin{split}
%   % Конец формулы будет на &
%   \text{Э}_\text{з.п} = \num\valTargetKPr \cdot
%     \bigl(\num\valTargetTBefore - \num\valTargetTAfter \bigr) \cdot
%     \num\valTargetTCh \cdot \num\valTargetNP \cdot
%     \biggl( & 1 + \frac{\num\valNdPercent}{100} \biggr) \times \\
%     \times
%     \biggl( 1 + \frac{\num\valNSotsPercent}{100} \biggr) =
%     \num\valEZP \rubFormula
%   \end{split}
% \end{equation}

% Экономическим эффектом является прирост чистой прибыли, полученной за счет экономии
% на текущих затратах предприятия, которое определяется по формуле:

% \begin{equation}
%   \label{eq:econ:targetDeltaPCh}
%   \Delta \text{П}_\text{ч} = \bigl(\text{Э}_\text{тек} -
%     \Delta \text{З}_\text{тек}^\text{п.с} \bigr)
%     \cdot \biggl( 1 - \frac{\text{Н}_\text{п}}{100} \biggr),
% \end{equation}
% \begin{explanationx}
%   \item[где] $ \text{Э}_\text{тек} $ -- экономия на текущих затратах при использовании программного средства, \rub;
%   \item $ \Delta \text{З}_\text{тек}^\text{п.с} $ -- прирост текущих затрат, связанных с использованием программного средства, \rub
% \end{explanationx}

% \FPeval{\valDeltaZTek}{0}
% \FPeval{\valETek}{\valEZP}
% \FPeval{\valTargetDeltaPCh}{round((\valETek - \valDeltaZTek) *
%   (1 - \valNPPercent / 100), \configRoundSigns)}

% Поскольку установка программного средства будет происходить на новых устройствах,
% то никаких дополнительных расходов по обновлению и установке не будет. Следовательно,
% $ \Delta \text{З}_\text{тек}^\text{п.с} = \num\valDeltaZTek \rubFormula $

% Источником экономии на текущих затратах является экономия на заработной плате
% сотрудников. Поэтому
% $ \text{Э}_\text{тек} = \text{Э}_\text{з.п} = \num\valETek \rubFormula $

% Используя определенные значения, найдем прирост чистой прибыли для
% организации-заказчика по формуле~(\ref{eq:econ:targetDeltaPCh}):

% \begin{equation}
%   \label{eq:econ:targetDeltaPChCalc}
%   \Delta \text{П}_\text{ч} = ( \num\valETek - \num\valDeltaZTek )
%     \biggl( 1 - \frac{\num\valNPPercent}{100} \biggr) =
%     \num\valTargetDeltaPCh \rubFormula
% \end{equation}

% \subsection{Расчет показателей экономической эффективности разработки и использования программного средства}

% \subsubsection{Расчет показателей экономической эффективности для организации-разработчика}

% Экономическая эффективность разработки для организации-разработчика выражается
% в значении нормы прибыли, которая определяется как отношение чистой прибыли
% к выручке по формуле:

% \begin{equation}
%   \label{eq:econ:Ri}
%   \text{Р}_\text{и} = \frac{\Delta \text{П}_\text{ч}}{\text{З}_\text{р}}
%     \cdot 100 \ \%.
% \end{equation}

% \FPeval{\valRi}{round(\valDeltaPCh / \valZr * 100, \configPercentRoundSigns)}

% Используя ранее определенные значения, найдем значение нормы прибыли
% по формуле~(\ref{eq:econ:Ri}):

% \begin{equation}
%   \label{eq:econ:RiCalc}
%   \text{Р}_\text{и} = \frac{\num\valDeltaPCh}{\num\valZr}
%     \cdot 100 = \num\valRi \ \%.
% \end{equation}

% \subsubsection{Расчет показателей экономической эффективности для организации-заказчика}

% Поскольку сумма инвестиций меньше суммы годового экономического эффекта,
% то оценка экономической эффективности инвестиций в разработку программного средства
% осуществляется с помощью расчета нормы прибыли по формуле:

% \begin{equation}
%   \label{eq:econ:targetRi}
%   \text{Р}_\text{и} = \frac{\Delta \text{П}_\text{ч}}{\text{Ц}_\text{пс}
%     \cdot \bigl( 1 + \frac{\text{Н}_\text{д.с}}{100} \bigr) }
%     \cdot 100 \ \%,
% \end{equation}
% \begin{explanationx}
%   \item[где] $ \text{Н}_\text{д.с} $ -- ставка налога на добавленную стоимость.
% \end{explanationx}

% \FPeval{\valNdsPercent}{20}
% \FPeval{\valTargetRi}{round(\valTargetDeltaPCh / (\valTsPs * (1 + \valNdsPercent
%   / 100)) * 100, \configPercentRoundSigns)}

% По состоянию на \econCalcDate, ставка налога на добавленную стоимость составляет
% $ \num\valNdsPercent \ \% $.

% Используя ранее найденные значения прироста чистой прибыли
% и цену программного средства, определим значение нормы прибыли
% по формуле~(\ref{eq:econ:targetRi}):

% \begin{equation}
%   \label{eq:econ:targetRiCalc}
%   \text{Р}_\text{и} = \frac{\num\valTargetDeltaPCh}{\num\valTsPs
%     \cdot \bigl( 1 + \frac{\num\valNdsPercent}{100} \bigr) }
%     \cdot 100 = \num\valTargetRi \ \%.
% \end{equation}

% \subsection{Вывод об экономической эффективности}

% В результате технико-экономического обоснования был произведен расчет инвестиций
% в разработку программного средства, расчет результата от разработки и использования
% программного средства и расчет показателей экономической эффективности разработки
% и использования. Общая сумма затрат на разработку программного средства составила
% $ \num\valZr $ рублей, отпускная цена -- $ \num\valTsPs $ рублей, чистая прибыль
% организации-разработчика -- $ \num\valDeltaPCh $ рублей, простая норма прибыли
% организации-разработчика -- $ \num\valRi \ \% $ и простая норма прибыли
% организации-заказчика -- $ \num\valTargetRi \ \% $.

% Анализируя полученные результаты, можно сделать однозначный вывод о том,
% что внедрение и использование данного программного средства в организации-заказчике
% оправдает первоначальные инвестиции в его разработку. С другой стороны,
% для организации-разработчика данного программного средства чистая прибыль,
% полученная от его разработки, будет полностью удовлетворять средние показатели
% рентабельности.

%\subsection{Вывод об экономической эффективности}
%
%В результате технико-экономического обоснования был произведен расчет инвестиций
%в разработку программного средства, расчет результата от разработки и использования
%программного средства и расчет показателей экономической эффективности разработки
%и использования. Общая сумма затрат на разработку программного средства составила
%$ \num\valZr $ рублей, отпускная цена -- $ \num\valTsPs $ рублей, чистая прибыль
%организации-разработчика -- $ \num\valDeltaPCh $ рублей, простая норма прибыли
%организации-разработчика -- $ \num\valRi \ \% $ и простая норма прибыли
%организации-заказчика -- $ \num\valTargetRi \ \% $.
%
%Анализируя полученные результаты, можно сделать однозначный вывод о том,
%что внедрение и использование данного программного средства в организации-заказчике
%оправдает первоначальные инвестиции в его разработку. С другой стороны,
%для организации-разработчика данного программного средства чистая прибыль,
%полученная от его разработки, будет полностью удовлетворять средние показатели
%рентабельности.
