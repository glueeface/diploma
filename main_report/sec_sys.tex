\section{СИСТЕМНОЕ ПРОЕКТИРОВАНИЕ}
\label{sec:sys}

В данном разделе описано разбиение проекта на модули. Это производится для того, чтобы программный продукт мог корректно работать на разных операционных системах, таких как:
\begin{itemize}
    \item Linux;
    \item Windows;
    \item FreeRTOS;
    \item TI-RTOS.
\end{itemize}

Выделение модулей также упрощает совместную разработку программного продукта несколькими программистами: каждый имеет возможность писать и тестировать отдельный модуль независимо от других. Также данный тип проектирования предотвращает возникновение ошибок на начальных этапах создания проекта, делает программный продукт более надежным.

После анализа требуемых для реализации программного продукта функций было решено разбить программу на следующие модули:

\begin{itemize}
    \item модуль \moduleCfg;
    \item модуль \moduleXml;
    \item модуль \moduleSettingsApply;
    \item модуль \moduleRecvPackets;
    \item модуль \moduleThreads;
    \item модуль \moduleProcessPackets;
    \item модуль \moduleDataStoring;
    \item модуль \moduleOsal;
    \item модуль \moduleLog.
\end{itemize}

Взаимосвязь между основными компонентами проекта отражена на структурной схеме
\structScheme.

\subsection{Модуль \moduleCfg}

Модуль \moduleCfg\ является самым важным модулем, который обеспечивает работу устройства. Одной из задач модуля является быстрая загрузка или восстановление работоспособности устройства после подачи питания. Он считывает данные из \cid-файла, который находится в файловой системе целевого устройства и содержит все базовые настройки оборудования РЗА. Модуль записывает считанные данные в память, откуда в свою очередь все модули системы берут стартовые значения при начальном запуске, экстренной перезагрузке или обновлении конфигурации.

\nomenclaturex{ICD}{IED Capability Description}{описание возможностей IED}
\nomenclaturex{XML}{Extensible Markup Language}{расширяемый язык разметки}

Считывание данных из \cid-файла происходит посредством модуля \moduleXml. При невозможности считать, обработать \cid-файл или его некоторые параметры, модуль \moduleCfg\ не дает системе прекратить свою работу. Для этого используются последние успешно загруженные настройки, которые сохраняются в энергонезависимой памяти. В случае первого запуска прибора или повреждении основного и резервного источников настроек, существует начальная конфигурация, поставляемая в качестве компонента программного обеспечения.

Описанный способ резервирования данных обеспечивает надежность системы.

\subsection{Модуль \moduleXml}

Данный модуль является вспомогательным для модуля \moduleCfg. Он считывает данных из \cid-файла и производит его обработку. \cid-файл использует \xml-разметку для \moduleDataStoring\ и имеет жесткую структуру.

\nomenclaturex{XSD}{\xml\ Schema Definition}{определение схемы \xml}

При считывании модуль \moduleXml\ проверяет соответствие данных, типов данных и внутренних структур с \xsd-шаблоном. Схемы проверки корректности \cid-файла являются общедоступными и разрабатываются организацией \iec. Использование \xsd-схем упрощает разработку и исключает необходимость проверки корректности данных разработчиками.
В случае обнаружения несоответствия с \xsd-схемой \cid-файл считается некорректным и его настройки не применяются.

Данный модуль используется только при добавлении новой схемы, поэтому не предъявляются требования, касаемые скорости работы и производительности.

\subsection{Модуль \moduleSettingsApply}

Основной задачей модуля \moduleSettingsApply\ является считывание данных из общей памяти, в которую модуль \moduleCfg\ с помощью модуля \moduleXml\ поместил их. Данный модуль записывает в остальные модули устройства необходимые данные для их дальнейшей корректной работы. При возникновении ошибки записи он повторяет свою попытку.

Проверка корректности записываемых данных также является функцией текущего модуля. При перезагрузке устройства, экстренном завершении или перезаписи конфигурации модуль \moduleSettingsApply\ должен регулировать процесс обновления настроек в каждом модуле, контролировать и предотвращать возникновение коллизий данных.

Одной из задач модуля является минимизация времени, в течение которого устройство не сможет принимать или обрабатывать входные данные. Исходя из этого к модулю \moduleSettingsApply\ предъявляются жесткие требования ко времени работы.

Данный модуль намеренно разделен с модулем \moduleCfg\ по следующим причинам:
\begin{itemize}
    \item разделение процесса \moduleSettingsApply\ на этапы и предъявление разных, независимых друг от друга требований к этим этапам;
    \item возможность распараллеливания разработки процесса \moduleSettingsApply;
    \item упрощение тестирования каждого модуля по отдельности;
    \item возможность разнесения модулей на разные исполняющие устройства и работу этих компонентов параллельно и независимо друг от друга. Особенно это актуально для систем жесткого и мягкого реального времени.
\end{itemize}

\subsection{Модуль \moduleRecvPackets}

Основной задачей модуля \moduleRecvPackets\ является получение Ethernet-пакетов с помощью модуля \moduleOsal. Рассматриваемый модуль отбрасывает пакеты, которые не соответствуют ожидаемому EtherType. Модуль \moduleRecvPackets\ обеспечивает фильтрацию по MAC-адресам, номерам VLAN, а также обеспечивает дедупликацию данных для протоколов HSR и PRP.

Поскольку модуль является первым модулем в системе, который занимается анализом принимаемых пакетов, его задачей является предотвращение обработки входных данных от тех источников, которые специально пытаются затормозить или прекратить работу устройства с помощью DoS-атак.

\nomenclaturex{DoS}{Denial of Service}{отказ в обслуживании}

В большинстве случаев GOOSE-пакеты передаются как мультикаст и широковещательные рассылки, что может привести к лавинному эффекту (широковещательному шторму). Во многих источниках возможность минимизации воздействия данного эффекта на сетевое оборудование называется предотвращением шторма и позволяет работать оборудованию в неблагоприятных сетевых условиях.

Если принятый пакет удовлетворяет текущей конфигурации системы, то модуль \moduleRecvPackets\ передает его для дальнейшего разбора модулю \moduleProcessPackets.

\subsection{Модуль \moduleThreads}

Система, элементом которой является данный дипломный проект, спроектирована таким образом, что возможна работа как на операционной системе Linux, так и на встраиваемых решениях с использованием систем жесткого и мягкого реального времени. Рассматриваемый модуль предназначен для динамического распределения обработки входящих пакетов с целью увеличения производительности и пропускной способности всей системы.

Особенностью реализации модуля является отслеживание поступающих на обработку потоков данных и обеспечения атомарности данных как в контексте одного потока, так и в контексте всей системы. Это может быть достигнуто с помощью использования примитивов синхронизации либо путем распределения нагрузки таким образом, чтобы в каждый момент времени любые два пакета, которые находятся в состоянии обработки, относились к различным потокам данных.

В случае использования систем реального времени возникает дополнительная сложность с переносом задач с одного вычислительного ядра на другое. Это особенно актуально для гетерогенных систем: систем на кристалле, которые содержат ядра различных архитектур и микроархитектур, поскольку нужно иметь версии приложения, предназначенные для каждой архитектуры.

\subsection{Модуль \moduleProcessPackets}

Модуль \moduleProcessPackets\ является вычислительной единицей, которой манипулирует модуль \moduleThreads. Каждый экземпляр данного модуля обрабатывает пакет данных, который был принят модулем \moduleRecvPackets.

Основными задачами этого модуля являются проверка корректности входящего пакета, извлечение необходимой информации, формирование структур данных и передача их в модуль \moduleDataStoring. При каждом обнаружении во входящих данных отклонений, превышающих допустимые нормы, модуль генерирует событие, которое может быть обработано другими модулями, и информация о возникновении которого будет записана в модуль \moduleLog.

Поскольку на данный модуль возлагается задача периодической обработки большого объема данных, он должен минимизировать время взаимодействия с каждым блоком. Это может быть сделано с помощью использования различных структур данных, которые обеспечивают эффективный доступ к последним по мере необходимости:
\begin{itemize}
    \item очереди приоритетов;
    \item кучи;
    \item хеш-таблицы;
    \item красно-черные деревья;
    \item префиксные деревья.
\end{itemize}

\subsection{Модуль \moduleDataStoring}

Модуль \moduleDataStoring\ отвечает за получение информации от модуля \moduleProcessPackets\ и сохранение ее в память. К модулю предъявляются особые требования к производительности, потому что новые пакеты приходят часто, данных много, а система должна с максимальной скоростью справляться с этой задачей. Также очень важно, чтобы данные не перемешивались, не повреждались, следовательно, все операции с памятью рекомендуется защищать примитивами синхронизации.

Модуль \moduleDataStoring\ не несет ответственность за передачу ему некорректной конфигурации для записи. За это отвечает модуль, который сохраняет данные, а рассматриваемый модуль лишь записывает то, что ему передано. Если часть модулей проявляют свою активность нерегулярно, то модуль \moduleDataStoring\ работает на протяжении всего жизненного цикла системы.

Также следует обратить внимание на то, что в процессе записи данные проходят проверку на целостность путем расчета контрольной суммы. Если каким-то образом поврежденная информация дошла до данного модуля, что практически исключается, то она не запишется в память.

Как было упомянуто выше, модуль безостановочно работает в процессе всего жизненного цикла системы, поэтому его реализация предполагается в отдельном потоке, параллельно с работой других модулей и независимо от них.

\subsection{Модуль \moduleOsal}


Для того, чтобы наша система имела возможность работать как на операционной системе Linux, так и на встраиваемых решениях с использованием операционных систем реального времени, существует модуль \moduleOsal. Он предоставляет общий интерфейс взаимодействия с различными операционными системами, поэтому каждая платформа имеет свою реализацию данного модуля. Это позволяет производить тестирование системы на базе одной операционной системы, а для остальных платформ требуется минимальное покрытие интеграционными тестами благодаря общей кодовой базе. Рассматриваемый модуль необходимо полноценно тестировать на каждой операционной системе по отдельности.

Модуль \moduleOsal\ является идеальным решением для параллельной поддержки нескольких операционных систем. Причем реализация одной и той же внешней функции производится на каждой платформе независимо друг от друга и могут разрабатываться одновременно.

\subsection{Модуль \moduleLog}

В процессе разработки и поддержки системы необходимо получать информацию о ее состоянии. Для этого используется модуль \moduleLog, с которым взаимодействуют некоторые модули. Он выводит информацию на экран или в среды передачи данных на всех этапах работы системы. Рассматриваемый модуль также сохраняет все сообщения, которые пишет система пользователям или разработчикам, в энергонезависимую память на внутренние или внешние носители. Место сохранения является настраиваемой опцией. Это сделано для того, чтобы существовала возможность в любой момент времени отследить ошибку, если она возникла. Также модуль \moduleLog\ может использоваться для отладки и тестирования, поскольку разработчику необходимо наблюдать реакцию системы при изменении алгоритма работы устройства.

Одной из задач модуля является форматирование и фильтрация входящих сообщений по уровням приоритета для каждого целевого носителя. Для любого уровня приоритета сообщения определен свой цвет в системе. Это сделано для удобства визуального восприятия и может быть отключено для части исходящих данных. Функция приоритезации по цветам особенно актуальна в средах передачи данных, в которых отсутствует обработка входящего потока информации на принимающей стороне.
