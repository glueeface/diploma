\section{СИСТЕМНОЕ ПРОЕКТИРОВАНИЕ}
\label{sec:sys}

%Системное проектирование – аналог разработки структурной схемы для аппаратных дипломных проектов.
%В этом разделе на основе системного подхода определяется обобщенная структура пакета программ, программы или модуля,
%описывается  назначение  выделенных  блоков,  а  также  связи  между  ними.
%Если программные средства разработки не заданы, то производится их выбор.
%Данный  раздел должен  сопровождать схему  структурную  либо схему  работы системы и содержать ссылки на нее.

В данном разделе описано разбиение проекта на модули.

Выделение модулей также упрощает совместную разработку программного продукта несколькими программистами: каждый имеет возможность писать и тестировать отдельный модуль независимо от других.
Также данный тип проектирования предотвращает возникновение ошибок на начальных этапах создания проекта, делает программный продукт более надежным.

После анализа требуемых для реализации программного продукта функций было решено разбить программу на следующие модули:

\begin{itemize}
    \item модуль \moduleAuth;
    \item модуль \moduleStripe;
    \item модуль \moduleTrading;
    \item модуль \moduleAutoTrading;
    \item модуль \moduleNotifications;
    \item модуль \moduleCommunicationPostgres;
    \item модуль \moduleCommunication;
    \item модуль \moduleParsing;
    \item модуль \moduleStatistics;
    \item модуль \moduleCommunicationMongoDB.
\end{itemize}

Взаимосвязь между основными компонентами проекта отражена на структурной схеме
\structScheme.

\subsection{Модуль \moduleAuth}\label{subsec:sys:module-auth}
Модуль \moduleAuth относится к основному микросервису, написанному с использованием фреймворков Django и Django Rest Framework.
Как понятно из названия -- модуль \moduleAuth будет отвечать за авторизацию и аутентификацию пользователей.
Он является основопологающим, так как будет обеспечивать доступ к определённому функционалу, обрабатывая каждый запрос к серверу.

Для обработки каждого запроса оптимальным вариантом реализации модуля \moduleAuth будет являтся реализация через промежуточное ПО,
про которое рассказывалось в пункте 1.3.3.
%~\ref{subsubsec:domain:middleware}.
Каждый запрос к серверу будет содержать базовый HTTP-заголовок Authentication, в котором будет находиться JWT.
Также данный модуль будет отвечать за подтверждение электронной почты пользователя посел регистрации.

Таким образом будет обеспечиваться один из принципов REST, а именно сервер не будет хранить состояние клиента.

\subsection{Модуль \moduleTrading}\label{subsec:sys:module-trading}
Данный модуль будет являтся основным, так как в нем будет определена вся бизнес-логика приложения.
Он будет отвечать за создание, обновление или удаление пользовательских офферов как на покупку, так и на продажу валюты.
Помимо операций покупок и продаж он также будет отвечать за создание пользовательского профиля и просмотра пользовательского портфеля.

Модуль \moduleTrading непосредственно связан с модулем ~\moduleCommunicationPostgres, так как результаты его работы будут сохраняться в базу данных.

Как и модуль \moduleAuth ~он будет реализован внутри основного микросервиса, написанного на Django и Django Rest Framework.

\subsection{Модуль \moduleAutoTrading}\label{subsec:sys:module-auto-trading}
Модуль \moduleAutoTrading ~является вспомогательным к модулю \moduleAutoTrading ~и отвечает за обработку запросов на покупку или продажу, которые должны выполниться автоматически.
Он будет периодически проверять все активные записи пользовательских офферов из базы данных, которые должны быть выполнены автоматически, и если условие покупки или продажи выполняются -- производить операцию.

Для реализации сервиса будут задействованы такие технологии, как Celery и Celery Beat.
Решение об использовании этих технологий было принято в связи с тем, что курсы валют постоянно изменяются, а следовательно после каждого обновления требуется проверять, не выполняются ли условия для произведения автоматической операции.
Такие действия будут времязатратными и Celery был разработан именно для времязатратных операций.
Celery Beat разработан для создания запронированных задач.
В данном проекте он будет отвечать за планирование периодических операций проверки условий для выполнения автоматических офферов.

\subsection{Модуль \moduleStripe}\label{subsec:sys:stripe}
Для выполнения операций на бирже пользователь должен пополнить свой счет.
Данный модуль будет отвечать именно за пополнение пользовательского счёта при помощи банковской карты прямо на сайте трейдинговой платформы.

Для реализации данного модуля было решено использовать платежную систему Stripe.
Из плюсов Stripe можно выделить предоставляемые платежные шлюзы, которые помимо кредитных и дебетовых карт позволяют принимать другие методы оплаты, такие как Apple Pay и Google Pay
Также Stripe обеспечивает высокий уровень безопасности, соответствуя стандартам PCI-DSS, предоставляя инструменты для защиты данных покупателей и предотвращения мошенничества.

    \nomenclaturex{PCI-DSS}{Payment Card Industry Data Security Standard}{стандарт безопасности индустрии платёжных карт}

\subsection{Модуль \moduleNotifications}\label{subsec:-module-notifications}
Данный модуль отвечает за уведомление пользователей об определённом событии, происходящем на бирже.

По умолчанию любые уведомления отключены, но если пользователь подтвердил свою электронную почту, то он сможет настроить уведомления на свой выбор.


\subsection{Модуль \moduleCommunicationPostgres}\label{subsec:sys:module-communication-postgres}
Данный модуль является связующим звеном основного микросервиса с базой данных PostgreSQL и основан на Django ORM.
Django ORM встроена в Django и относится к типу Active Record, однако она была выбрана по следующим причинам:
\begin{itemize}
    \item Односторонняя связь.
    Каждый объект в приложении представляет одну запись в базе данных.
    Объекты могут быть легко созданы, изменены и удалены, и эти операции отражаются на соответствующих записях в базе данных.
    \item Прозрачность доступа к данным.
    Active Record скрывает сложности взаимодействия с базой данных и предоставляет простой интерфейс для работы с данными.
    Это позволяет сосредоточиться на бизнес-логике приложения, а не на деталях работы с базой данных.
    \item Интеграция с бизнес-логикой.
    Active Record позволяет добавлять методы и поведение непосредственно к моделям данных.
    Это делает модели активными участниками бизнес-логики приложения и упрощает создание сложных операций над данными.
\end{itemize}

Помимо этого Django ORM использует механизм миграций, который позволяет управлять структурой базы данных через код.
%Это делает процесс развертывания и обновления базы данных более управляемым и автоматизированным.

\subsection{Модуль \moduleCommunication}\label{subsec:sys:module-communication}
Данный модуль позволяет осуществлять передачу информации между двумя микросервисами и построен на брокере сообщений Apache Kafka.
Он будет осуществлять передачу в обе стороны.
Данный способ передачи информации был выбран в силу его асинхронности.

Со стороны основного микросервиса он будет доставлять информацию о новых валютах, криптовалютах или акциях, которые были добавлены в систему.
%Это действие будет привязано к событию доб
Со стороны микросервиса, написанного на FastAPI он будет периодически доставлять информацию о валютах, криптовалютах или акциях.

Как понятно из названия, данный модуль будет связывать оба микросервиса.
И в силу модели работы Apache Kafka будет иметь как минимум одного производителя и потребителя на каждом из микросервисов.

\subsection{Модуль \moduleParsing}
Модуль \moduleParsing относится к микросервису, написанному на FastAPI и служит для получения информации о курсах валют, криптовалют и акций.
Он непосредственно связан с модулем \moduleCommunicationMongoDB, при помощи которого сохраняет информацию о курсах валют в базу данных.
Он также связан с модулем \moduleCommunication, при помощи которого происходит передача последней полученной информации из базы данных на основной микросервис.

\subsection{Модуль \moduleStatistics}
Модуль статистики, как и модуль \moduleParsing относится к микросервису, написанному на FastAPI.
Данный модуль служит для предоставления пользователям различного вида статистики и её визуализации посредством графиков.
Он непосредственно связан с модулем \moduleCommunicationMongoDB, так как информация,
требуемая для предоставления статистики и создания графиков, хранится в СУБД MongoDB.

\subsection{Модуль \moduleCommunicationMongoDB}
Данный модуль связывает микросервис, написанный на FastAPI и СУБД MongoDB.
Он основан на ассинхронном движке motor, так как сам FastAPI основан на ассинхронной модели взаимодействия клиента и сервера.
Помимо этого он позволит ускорить операции взаимодействия сервера и СУБД.
